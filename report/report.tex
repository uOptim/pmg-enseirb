\documentclass{article}

\usepackage[french]{babel}
\usepackage[utf8]{inputenc}
\usepackage[T1]{fontenc}
\usepackage{verbatim}
\usepackage{graphicx}
\usepackage{calc}
\usepackage{color}
\usepackage{float}
\usepackage{hyperref}
\usepackage{geometry}

\geometry{margin=2cm}

\title{Projet système \\ Mise en place d'une bibliothèque de threads}
\author{Benoît Ruelle, Ludovic Hofer}

\begin{document}
\begin{center}
  \includegraphics [width=40mm]{ENSEIRB-MATMECA.jpg} 

\vspace{\stretch{1}}

\textsc{\Huge Simulation de particules sur GPU}\\[0.5cm]
\rule{0.4\textwidth}{1pt}

\vspace{\stretch{1}}

\begin{center}
  
  \begin{flushleft}
    \large
    \emph{Auteurs :}\\
    \begin{itemize}
    \item Benoît Ruelle
    \item Ludovic Hofer
    \end{itemize}
  \end{flushleft}
  
  
  \begin{flushright}
    \large
    \emph{Encadrant :}\\
    Raymond Namyst
  \end{flushright}
\end{center}

\vspace{\stretch{1}}

{\large \url{https://github.com/uOptim/pmg-enseirb/}}

\vspace{\stretch{1}}

{\large Deuxième année, filière informatique}

~

{\large 14 avril 2013 - 6 juin 2013}\\

\end{center}
\thispagestyle{empty}
\pagebreak
\tableofcontents
\newpage

\section{Introduction}

\section{Développement et difficultés rencontrées}

\subsection{Division par zéro dans le cas des collisions}
Blabla à propos de l'équation qui donne une division par zéro

\subsection{Mise en place de la mesure}
Blabla à propos de l'emplacement de la mesure de vsync etc.

\subsection{Collision v3}
Intro à colli v3
\subsubsection{Répartition des workgroups/threads}
\subsubsection{Calcul de la position en fonction du group id}
\subsubsection{Complexité}
Nombre de threads élevé, mais calcul de la racine carrée... à voir si en
changeant la slice on améliore les résultats
\subsubsection{En faisant varier la slice}
Tests à faire..

\subsection{Lennard Jones v2}

\subsection{Génération de fichiers pour les atomes}
Blabla sur le fichier de génération python : espace grand pour éviter la
diffusion des atomes sur lennard Jones

\section{Performances mesurées}

\subsection{Cas de la fonction collision}
Graphiques + commentaires
\subsection{Cas du calcul des forces de Lennard-Jones}
Graphiques + commentaires

\section{Conclusion}

\end{document}
